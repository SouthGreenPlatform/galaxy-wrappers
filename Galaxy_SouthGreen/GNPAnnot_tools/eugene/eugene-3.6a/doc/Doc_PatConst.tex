% Documentation of the PatConst sensor

\subsubsection{\texttt{Sensor.PatConst}}

\paragraph{Description}

This plugin predicts signals at each occurence of a pattern on the
sequence. The corresponding uniform costs for using
or rejecting a signal can be set using the
\texttt{PatConst.patP*[i]}/ and \texttt{PatConst.patPNo*[i]} parameters.

The sensor is activated by setting the value 1 (one instance of the
plugin) or an integer (i instance) for the parameter
\texttt{Sensor.PatConst.use} in the parameter file.

Here is an example of PatConst parameters definition (2 instances) :
\begin{Verbatim}[fontsize=\small]
PatConst.type[0]        donor    # Possible types : start insertion deletion
PatConst.pat[0]         GC       #  transstart transstop stop acceptor donor
PatConst.newStatePos[0] 1        # Position of the new state in the pattern
PatConst.patP*[0]       -25
PatConst.patPNo*[0]     0
#
PatConst.type[1]        acceptor
PatConst.pat[1]         AG   
PatConst.newStatePos[1] 3
PatConst.patP*[1]       -40
PatConst.patPNo*[1]     0
#
Sensor.PatConst.use     2
Sensor.PatConst         1        # Sensor priority
\end{Verbatim}

\paragraph{Input files format}

No file input.

\paragraph{Integration of information}

All predictions that use a predicted signal receive the
corresponding \texttt{PatConst\-.patP*[i]} penalty while those that go through
a predicted splice site while they could have used it receive a
\texttt{PatConst.patPNo*[i]} penalty.

\paragraph{Post analyse}

No post analyse.

\paragraph{Graph}

No plot.
